\chapter{Paper} \label{c:Paper}
In this chapter we will examine the topic of the paper \textit{Stable Neo-Hookean Flesh Simulation} \cite{Smith:2018:SNF:3191713.3180491}. In the interest of understanding the thought process of the authors I will include some of their calculations a bit more detailed. In addition examples and visualisations should help for a better perception. 

\section{Deformation Gradient}

\begin{table}[!htbp]
\centering
    \begin{tabular}{ | l | l |}
    \hline
    \textbf{Symbol} & \textbf{Definition} \\ \hline
    $\mathrm{F=RS}$ & Polar decomposition \\ \hline
    $J=\operatorname{det}(\mathrm{F})$ & Relative volume change \\ \hline
    $\mathrm{C}=\mathrm{F}^{T} \mathrm{F}$ & Right Cauchy-Green  \\ \hline	
    $I_{C}=\operatorname{tr}(\mathrm{C})$ & First right Cauchy-Green invariant \\ \hline
    \end{tabular}
    \caption{Quantities Derived from the Deformation Gradient \textbf{F} taken from \cite{Smith:2018:SNF:3191713.3180491}}
\label{table:1}
\end{table}

\todoredefined[inline]{
TODO: Add in Background? Explain more each entity
}


\section{Energy Formulation}

\subsection{Stability}
The core goal of the paper was to model deformations for virtual characters that have human-like features. They concentrated on the deformation energy. In order to achieve a convincing result as a first step we need to specify some requirements. For our needs in this case the stability of the energy is important. More precisely we need a hyperelastic energy that is stable in the following four important ways:

\textbf{1. Inversion Stability:} Given some arbitrary object it is possible that while deforming the object we can arrive at a zero volume state or even an entire inversion. Take for example the tetrahedron shown in figure \ref{fig:inversion_1}. In figure \ref{fig:inversion_2} we see a deformed state of this tetrahedron where the volume is scaled down to zero and we are left with a simple triangle. In figure \ref{fig:inversion_3} image we have an inversion of the object. The needed deformation energy has to be able to deal with both cases. That means that the energy has to be singularity-free and does not need any filters or threshold (\cite{Smith:2018:SNF:3191713.3180491}, 12:3).

\todoredefined[inline]{
TODO: Explain what last sentence means. Reference correct like this?
}

\begin{figure}[!ht]
\centering
\begin{subfigure}{.3\textwidth}
  \centering
  % include first image
  \includegraphics[width=.8\linewidth]{resources/inversion_1}  
  \caption{Rest state}
  \label{fig:inversion_1}
\end{subfigure}
\begin{subfigure}{.3\textwidth}
  \centering
  % include first image
  \includegraphics[width=.8\linewidth]{resources/inversion_2}  
  \caption{Zero-volume state}
  \label{fig:inversion_2}
\end{subfigure}
\begin{subfigure}{.3\textwidth}
  \centering
  % include second image
  \includegraphics[width=.8\linewidth]{resources/inversion_3}  
  \caption{Inversed state}
  \label{fig:inversion_3}
\end{subfigure}
\caption{Inversion of a tetrahedron {\cite{STREAM2018}}}
\label{fig:inversion}
\end{figure}


\textbf{2. Reflection stability:} A reflection is rotation around the coordinate origin.
For the deformation energy we need it to be well behaved regardless of the reflection convention used in the singular value decomposition.

\textbf{3. Rest stability:} When deforming an object in a certain way we apply one or multiple forces over that object. With rest stability we want that if the sum of forces is equal to zero the object must be back in its rest state.

\textbf{4. Meta-stability under degeneracy:} We can crush an object into an arbitrary shape like a plane, line or point. That process is illustrated for a cube in figure \ref{fig:meta_stability}. The cube should now be able to recover to its actual shape after the deformation.

\begin{figure}[!htbp]
	\centering
	\includegraphics[width=0.5\textwidth]{resources/meta_stability}
	\caption{Illustration of meta stability {\cite{STREAM2018}}}
	\label{fig:meta_stability}
\end{figure}

Based on these four requirements we will in the following determine if a deformation energy is suited for our needs.

\todoredefined[inline]{
TODO: Finish reflection stability. Maybe add own images? Some more sources?
}

\subsection{Existing Neo-Hookean Energies}
A commonly used energy function is defined as follows (see e.g. \cite{bonet1997nonlinear}, p.148):

\[
\Psi=\frac{\mu}{2}\left(I_{C}-3\right)-\mu \ln J+\frac{\lambda}{2}(\ln J)^{2}
\]

where $\lambda$ and $\mu$ are the material constants and $J = det(F)$.

In the literature appear many more different formulations of the energy. Each one consists of a 1D length term and a 3D volume term.

\textbf{1D Length Term:} This term penalizes length changes.

\textbf{3D Volume Term:} Here we are dealing with a volume-preserving penalty term. 

\subsection{Rest Stabilization}

\subsection{Lamé Reparametrization}

\subsection{Previous Work}
\todoredefined[inline]{
TODO: Here comes previous work in neo-hookean energy formulation. What is neo-hookean and why do we need it here?
And what is wrong with each one.
}


\subsection{Stable Neo-Hookean Energy}
Conclude to the energy proposed in the paper.

\section{Energy Analysis}
Calculations and Herleitungen

\subsection{First Piola-Kirchhoff Stress (PK1)}
Explain.

\subsection{The Energy Hessian Terms}
Calculations

\subsection{The Tikhonov, Mu, and Gradient Terms}
Calculations

\subsection{The Volume Hessian}
Calculations
 
\subsection{The Complete Eigensystem}
Calculations

\section{Experiments with the Code}
The authors of the paper \textit{Stable Neo-Hookean Flesh Simulation} \cite{Smith:2018:SNF:3191713.3180491} kindly provided the implementation for an application of their formulated energy. In this code they implemented the stretch test on a cube. The outputs were 26 static images with show the deformation in 25 steps. The parameters the code expects are the two lamé parameters $\mu$ and $\lambda$ and a value for defining the desired resolution the objects should have.

\todoredefined[inline]{
TODO: Explain how the code is implemented in simple words and how the energy is taken in account with the poisson's ratio. Make scheme with input - box / code - output. Do I have to reference code?
}

For starters let us take common values for $\mu$ and $\lambda$. We first start with $\mu = 1.0$, $\lambda = 10.0$ and a resolution of 10.0. For the poisson's ratio we get the value $0.4545$:

\[ \sigma =  \frac{10.0}{2 (10.0 + 1.0)} = 0.4545 \in [-1, 0.5] \]


The following images in figure \ref{fig:stretchtest} show the stretch test with $\mu = 1.0$, $\lambda = 10.0$ and a resolution of 10.0 on a tetrahedral and a hexahedral mesh.

\begin{figure}[!htbp]
	\centering
	\begin{subfigure}[b]{\textwidth}
        \centering
        \includegraphics[width=0.24\textwidth]{resources/hexcli_step0.png}
        \hfill
        \includegraphics[width=0.24\textwidth]{resources/hexcli_step8.png}
        \hfill
        \includegraphics[width=0.24\textwidth]{resources/hexcli_step16.png}
        \hfill
        \includegraphics[width=0.24\textwidth]{resources/hexcli_step24.png}
        \caption{Stretch test on a hexahedral mesh}
    \end{subfigure}
    \vskip\baselineskip
    \begin{subfigure}[b]{\textwidth}
        \centering
        \includegraphics[width=0.24\textwidth]{resources/tetcli_step0.png}
        \hfill
        \includegraphics[width=0.24\textwidth]{resources/tetcli_step8.png}
        \hfill
        \includegraphics[width=0.24\textwidth]{resources/tetcli_step16.png}
        \hfill
        \includegraphics[width=0.24\textwidth]{resources/tetcli_step24.png}
        \caption{Stretch test on a tetrahedral mesh}
    \end{subfigure}
    \caption{Stretch test performed on a cube with (a) a hexahedral mesh and (b) a tetrahedral mesh}
    \label{fig:stretchtest}
\end{figure}


\todoredefined[inline]{
TODO: Load into OpenFlipper and screenshot results. Include more examples and what went right and what went wrong.
}



\section{Discussion}
Stuff, Taylor approx.


