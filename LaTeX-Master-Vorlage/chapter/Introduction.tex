\chapter{Introduction}
\textit{"Animation offers a medium of story telling and visual entertainment which can bring pleasure and information to people of all ages everywhere in the world."} \\
- Walt Disney


\section{Motivation}
With steadily increasing computational power the demand of better results is constantly increasing. Especially in the field of animation and simulation we are no longer happy with mediocre results. 
In the entertainment sector the gaming industry and animation studios like Pixar\textsuperscript{\textcopyright} or Disney\textsuperscript{\textcopyright} brought us games and movies of highest quality. Both of them have made groundbreaking progress over the years. This is easily seen when we compare today's work with that ten years ago. 

As always we have different requirements for each use.
In some cases we want to exaggerate a movement or a reaction in a certain way. We can for example create a massive explosion in a movie that would not be half as spectacular in the real world. 

In other scenarios we want to come as close as possible to reality. For instance we want an animated character to move and physically interact with its environment as a real human being would. Otherwise our brain would immediately recognize that some things do not add up.
The goal here is to bring characters quite literally to life. We can add small details like visible breathing and small wrinkles to have an even more convincing effect. The goal is to create the illusion of a character with personality, thought and emotions. In order to achieve this effect we need the character to move and react nearly physically correct. 

In the paper \textit{Stable Neo-Hookean Flesh Simulation} \cite{Smith:2018:SNF:3191713.3180491} the authors addressed exactly the problem of making an animated movement of a human-like character look as natural as possible. In order to animate a physical movement we need to understand the physics behind it which lies in the field of continuum mechanics. Unfortunately most of the time it has yet to be learned. The goal of this thesis is for a regular computer science student to give the necessary physical and mathematical background to understand the field of animation and maybe make a contribution to the field.


\section{Structure}
Following up I will give a brief overview of the necessary mathematical background and deliver an introduction in continuum mechanics. Next up I will go through the ideas made in the paper mentioned and include some calculations and visualisations that help for a better understanding. \\


\todoredefined[inline]{
TODO: Adjust the introduction according to additions in text. Improve quote at beginning. Maybe add some images taken from Incredibles 2 for better visualisation?
}