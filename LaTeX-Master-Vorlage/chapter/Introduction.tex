\chapter{Introduction}
\textit{``Animation offers a medium of story telling and visual entertainment which can bring pleasure and information to people of all ages everywhere in the world.``} \\
- Walt Disney


\section{Motivation}
With steadily increasing computational power, the demand of better results is constantly growing. Especially in the field of animation and simulation we are no longer happy with mediocre results. 
In the entertainment sector and gaming industry animation studios like Pixar\textsuperscript{\textcopyright} or Disney\textsuperscript{\textcopyright} brought us games and movies of highest quality. Both of them have made groundbreaking progress over the years. This is easily observed when we compare today's work with that from ten years ago. 

As always, we have different requirements for each use.
In some cases we want to exaggerate a movement or a reaction in a certain way. We can for example create a massive explosion in a movie that would not be half as spectacular in the real world. 

In other scenarios we want to come as close as possible to reality. For instance, we may want an animated character to move and physically interact with its environment as a real human being would. Otherwise, the human brain would immediately recognize that some things do not add up.
The goal here is to bring characters quite literally to life. We can add small details like visible breathing and small wrinkles to have an even more convincing effect. We aim to create the illusion of a character with personality, thought and emotions. In order to achieve this effect, we need the character to move and react physically correct. 

In the paper \textit{\acrlong{snh}} \cite{Smith:2018:SNF:3191713.3180491}, the authors addressed exactly this problem of making an animated movement of a human-like character look as natural as possible. Consequently this thesis is based heavily on this paper. In the following, I will abbreviate the name of the paper with \textit{\acrshort{snh}}. \\
But before diving further into the content of the paper, a fundamental background is needed. In order to animate a physical movement, we first need to understand the physics behind it which requires some knowledge in the field of continuum mechanics. Unfortunately, for most of us it has yet to be learned. The goal of this thesis is to give the necessary physical and mathematical background for a regular computer science student to understand the field of animation. In addition I will go deeper into the thematics of the paper \textit{\acrshort {snh}}. I aim to get an understanding of their contribution in the field and implement their proposed energy myself.


\section{Structure}
In the following, I will give a brief overview of the necessary mathematical background and deliver an introduction into the field of continuum mechanics. Next, I will go through the ideas mentioned in the paper and include some calculations and visualisations that serve for a better understanding. Lastly I will give an insight into the process of implementing the energy. \\


\todoredefined[inline]{
TODO: Adjust according to additions in text.
}