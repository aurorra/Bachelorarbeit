\chapter{Introduction}
\label{c:Introduction}
\textit{``Animation offers a medium of storytelling and visual entertainment which can bring pleasure and information to people of all ages everywhere in the world.''} \\
- Walt Disney

\section{Motivation}
With steadily increasing computational power, the demand for better-looking results is constantly growing. Especially in the field of animation and simulation, various research has been done to avoid mediocre results. In the entertainment sector, these findings were used to present movies of the highest quality for people all over the world. Animation studios like Pixar\textsuperscript{\textcopyright} have made groundbreaking progress over the years. This can easily be observed by comparing today's work with that from ten years ago. 

As always, there are different requirements for each use case. In some cases, an exaggerated movement or reaction has a better effect on the viewer. A massive explosion, for example, looks more spectacular in a movie when it is not physically correct. In other scenarios, the animation should come as close as possible to reality. For instance, adding small details like visible breathing or wrinkles offers a way to get more convincing results. A good animation should create the illusion of a character with personality, thoughts, and emotions. But in order to create this illusion, an animated character has to move and physically interact with its environment as it would in reality. Otherwise, the human brain would immediately recognize that some things do not add up. One aspect for achieving this realistic effect is the simulation of the fleshy appearances of virtual characters. In the paper \textit{\acrlong{snh}} (\cite{Smith:2018:SNF:3191713.3180491}), the authors addressed exactly this problem and formulated a new deformation energy for simulating human flesh. This thesis is based heavily on this paper. In the following, I will abbreviate the name of the paper with \textit{\acrshort{snh}}.

But before diving further into the content of the paper, a fundamental background is needed. In order to animate a physical movement, a basic understanding of the physics behind it is necessary. This knowledge requires some experience in the field of continuum mechanics. Unfortunately, for most of the students in computer science, it has yet to be learned. This thesis should help students of computer science who are interested in the field of animation and simulation but are still beginners. Therefore, the goal of this thesis is to explain the necessary physical and mathematical background. In addition, I will go deeper into the thematics of the paper \textit{\acrshort {snh}}. I will go through some of their calculations more detailed to help the reader to follow the thought process of the authors.
I aim to get an understanding of their contribution to the field and extend the code they provided to test the energy with practical experiments.

\section{Structure}
In the following, I will deliver an introduction into the field of continuum mechanics and give a brief overview of the necessary mathematical background. Next, I will go through the ideas mentioned in the paper \acrshort{snh} and include some calculations and visualizations that serve for a better understanding. Lastly, I will give insights into the implementation of the energy and show the results of the practical experiments I conducted.